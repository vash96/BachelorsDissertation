\chapter{Conclusioni}

È stato presentato il Commesso Viaggiatore Simmetrico come uno dei problemi fondamentali 
dell'Ottimizzazione Combinatoria. La \textit{difficoltà} che richiede la ricerca della soluzione 
ottima e l'importanza che tale soluzione ha in svariati ambiti applicativi ci ha spinto 
a trattare il problema dal punto di vista degli algoritmi approssimati. Abbiamo presentato alcune 
delle più famose euristiche di tipo ``tour construction'', ``tour improvement'' e ``composite''. 
La famiglia ``costruttiva'' si compone di algoritmi che definiscono un tour valido vertice dopo vertice. 
La famiglia ``migliorativa'' comprende euristiche che analizzano un tour valido e cercano di modificarlo 
per ottenere un tour valido migliore. L'ultima classe si compone di algoritmi che tentano di combinare i 
vantaggi di entrambe. Abbiamo poi implementato gli algoritmi descritti e trattato, dal punto di vista empirico, 
le caratteristiche che riteniamo fondamentali in un algoritmo approssimato: velocità di esecuzione e 
errore rispetto all'ottimo; dai risultati sperimentali è emersa una correlazione tra queste due 
proprietà. Come ci si aspettava, euristiche meno elaborate impiegano meno tempo per trovare il tour 
``quasi''-ottimo, al costo di avere un elevato errore; viceversa, euristiche più elaborate impiegano 
un tempo notevolmente maggiore per trovare una soluzione, che sarà di qualità migliore. La scelta 
dell'euristica da utilizzare, tuttavia, dipende strettamente da quale caratteristica si voglia 
prediligere.