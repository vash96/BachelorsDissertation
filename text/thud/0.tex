\chapter*{Introduzione}

Il Problema del Commesso Viaggiatore consiste nella ricerca del Circuito Hamiltoniano
di costo minimo in un grafo pesato. Formalmente, dato un grafo $G=(V,E)$ e una funzione di peso
$w: E \rightarrow \mathbb{N}$ siamo interessati ad una permutazione dei vertici del grafo
$(v_{i_1}, v_{i_2}, \dots, v_{i_{|V|}})$ che minimizzi la quantità
$$W = w(v_{i_{|V|}}, v_{i_1}) \;+ \displaystyle\sum_{j=1}^{|V|-1}{w(v_{i_j}, v_{i_{j+1}})}$$

Nell'ambito della Ricerca Operativa e dell'Ottimizzazione
Combinatoria, l'importanza del TSP (dall'inglese \textit{Traveling Salesman Problem}) deriva dal 
fatto di poterlo utilizzare per descrivere un ricco insieme di \textit{real world problems} riguardo, 
ad esempio, Cristallografia, \textit{Vehicle routing}, \textit{Scheduling}, Robotica, ecc...

Dal punto di vista teorico, la versione decisionale di TSP è notoriamente NP-Completa e ciò lo
rende, nella pratica, intrattabile anche per piccole istanze del problema.
Dalla necessità di risolvere il TSP e dall'impossibilità (assumendo $P\neq{}NP$) di poterlo
fare efficientemente per una qualsiasi istanza del problema, nasce il concetto di ``Euristica'', ovvero una tecnica che ha
lo scopo di trovare in un tempo ragionevole un'approssimazione sufficientemente buona della soluzione esatta.
\ \\
Nell'ambito del TSP possiamo dividere le euristiche in tre classi:
\begin{itemize}
    \item Tour construction: l'algoritmo costruisce il circuito aggiungendo, passo dopo passo, un nuovo
            vertice a seconda di una strategia più o meno elaborata
    %
    \item Tour improvement: a partire da un circuito arbitrario, l'algoritmo esegue delle ``mosse'' che hanno
            lo scopo di migliorarlo
    %
    \item Composite: vengono combinate le tecniche di costruzione e di miglioramento
\end{itemize}

In questa tesi studieremo il Commesso Viaggiatore Simmetrico (STSP), ovvero assumeremo
che il grafo sia non diretto, completo e con funzione di peso simmetrica. Presenteremo delle euristiche
appartenenti ad ognuna delle classi e ne analizzeremo le performance sia dal punto di vista dell'efficienza
in tempo, sia dal punto di vista dell'efficacia nel senso della bontà della soluzione trovata rispetto all'ottimo.
Gli algoritmi scelti sono Nearest Neighbor, Nearest Insertion, Christofides, 3-OPT (in tre versioni) e 
Lin-Kernighan-Helsgaun: per quanto riguarda Christofides e Lin-Kernighan-Helsgaun si è deciso di implementare 
una versione più semplice degli algoritmi in letteratura; possiamo quindi dire che le euristiche implementate 
siano solamente ispirate a quelle descritte in \cite{CHR}, \cite{LKH}; per quanto riguarda 3-OPT, 
si è scelto di implementare la versione descritta in \cite{3opt}.

I test sono stati effettuati su un insieme di grafi selezionati da TSPLIB\cite{tsplib} e su un insieme di grafi generati
casualmente. Le misurazioni sono state effettuate seguendo le procedure descritte in \cite{Poli}. Tutti i programmi sono 
stati scritti in \texttt{C++}.
